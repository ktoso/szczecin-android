% \documentclass{beamer}
% \usetheme{default} 
% 
% \setbeamercolor{structure}{fg=green!40!black} 
% \usebackgroundtemplate{
%     \centering\includegraphics[width=\paperwidth,height=\paperheight]{images/android_wall}
% }
% \setbeamertemplate{navigation symbols}{}
% 
% \usepackage[polish]{babel}
% \usepackage[utf8x]{inputenc}
% \usepackage{t1enc}
% \usepackage{default}
% \usepackage{listings}
% 
% \lstset{language=java, basicstyle=\small, commentstyle=\color{gray}}
% \lstset{frame=single}
% 
% \usefoottemplate{
%   \vbox{
%     \tinycolouredline{structure!25}{
%       \color{black}\textbf{	
%         \insertshortauthor\hfill
%         Android @ Szczecin 2011
%       } 
%     }
% %    \tinycolouredline{structure}{
% %      \color{white}\textbf{\insertshorttitle}\hfill
% %    } 
%   }
% }
% 
% \title{Android @ Szczecin \\ 2011}
% \author{Konrad Malawski \\ konrad.malawski@java.pl}
% 
% \begin{document}


\begin{frame}[fragile]\frametitle{Android Logger}
Logujemy przy pomocy \textbf{android.util.Log}.\\
Korzystamy z niego następująco:

\begin{lstlisting}
class MyClass {
  private final String TAG = getClass()
                             .getSimpleName();

  void doStuff() {
    Log.d(TAG, "doing stuff..."); // debug
  }
}
\end{lstlisting}
\end{frame}

\begin{frame}[fragile]\frametitle{Android Log levels}
Dostępne poziomy logowania to:
\begin{itemize}
 \item \verb|Log.v()| - VERBOSE
 \item \verb|Log.d()| - DEBUG
 \pause \item \verb|Log.i()| - INFO
 \pause \item \verb|Log.w()| - WARN
 \pause \item \verb|Log.e()| - ERROR
 \pause \item \verb|Log.wtf()| \pause - \textbf{W}hat a \textbf{T}errible \textbf{F}ailure
\end{itemize}

\end{frame}


\begin{frame}\frametitle{Sposoby przechowywania danych}
\begin{itemize}
% http://developer.android.com/guide/topics/data/data-storage.html
 \item  \textbf{Shared Preferences} - Prosty KeyValue store, idealny dla prostych ustawień aplikacji etc
 \pause \item \textbf{Internal Storage} - Zapisywanie plików w swoim formacie na \textbf{wewnątrznej pamięci} urządzenia (dobre dla cache obrazków etc)
 \pause \item \textbf{External Storage} - Zapisywanie plików w swoim formacie na np. \textbf{karcie SD} (dobre dla cache obrazków etc)
 \pause \item \textbf{SQLite Database} - Zwyczajna instancja bazy danych SQLite - m.in. tym sposobem dostajemy informacje o kontaktach
 \pause \item \textbf{Cloud Storage} - Nie trzymamy danych lokalnie, pchamy i pobieramy wszystko z chmurki
\end{itemize}
\end{frame}


\begin{frame}\frametitle{Shared Preferences}
\begin{center}
 \Huge{Shared Preferences} 
\end{center}
\end{frame}

\begin{frame}[fragile]\frametitle{SharedPreferences - API}
Uzyskanie instancji:
\begin{lstlisting}
// zawsze zadziala:
Context ctx = getApplicationContext(); 

// w Activity lub Service jest latwiej:
// ctx = this;

SharedPreferences prefs = PreferenceManager
                 .getDefaultSharedPreferences(ctx);
\end{lstlisting}
\end{frame}


\begin{frame}[fragile]\frametitle{SharedPreferences - Read API}
 Przykład \textbf{odczytania} zmiennej:
\begin{lstlisting}
 // oto jak korzystac z @strings/ w Activity
 String key = getString(R.string.key_sound_notif);

 String s = preferences.getString(key, "undefined");
 
 // analogicznie dla Integer/Long/Double/StringSet
\end{lstlisting}
\end{frame}

\begin{frame}[fragile]\frametitle{SharedPreferences - Write API}
 Przykład \textbf{zapisania} zmiennej:
\begin{lstlisting}
SharedPreferences.Editor editor = preferences.edit();

editor.putString(key, "new value");
// analogicznie dla Integer/StringSet/Double ...

editor.commit();
\end{lstlisting}
\end{frame}


\begin{frame}[fragile]\frametitle{Shared Preferences}
 Shared w sensie ,,wewnątrz \textbf{naszej} aplikacji'', nie między wieloma.\\
 SharedPreferences zapisywane są w: \\
 \begin{center}
  \texttt{/data/data/pl.project13.myapp/shared\_prefs} 
 \end{center}

\pause

Można się tam dobrać gdy się jest \textbf{root}:
\begin{verbatim}
$ abd shell
> cat /data/data/pl.project13.myapp/shared_prefs

<map>
 <string name="pass">zomg_its_plain_text</string>
 <!-- ... -->
</map>
\end{verbatim}
\end{frame}

% SQL kontakty http://developer.android.com/guide/topics/providers/content-providers.html
% todo można by tutaj jeszcze o SQL i poszukiwaniu kontaktów

\begin{frame}\frametitle{Security a SharedPreferences}
\begin{center}
 Wniosek jest prosty:\\ 
 Szyfrujemy ważne rzeczy trzymane gdziekolwiek na komórce.
\end{center}
\end{frame}


\begin{frame}\frametitle{Robo Guice - Google Guice for Android}
 \begin{figure}[c]
 \includegraphics[width=0.5\textwidth]{images/roboguice}  
 \end{figure}
 
 \begin{center}
  Google Guice = \textbf{JSR-330} Dependency Injection for Java
 \end{center}
\end{frame}

\begin{frame}[fragile]\frametitle{RoboGuice - co zyskujemy?}
Przed:
\begin{lstlisting}
class Act extends Activity {
 SharedPreferences prefs;
 EditText mLogin;
 public void onCreate(Bundle savedState) {
  prefs = PreferenceManager
          .getDefaultSharedPreferences(this);
  mLogin = (EditText) findById(R.id.login);
 }
}
\end{lstlisting}

\pause

Po:
\begin{lstlisting}
class Act extends RoboActivity {
 @Inject SharedPreferences prefs;
 @InjectView(R.id.login) EditText mLogin;
 public void onCreate(Bundle savedState){ /**/ }
}
\end{lstlisting}

\end{frame}


\begin{frame}[fragile]\frametitle{Zapięcie RoboGuice w 4 krokach:}
\begin{enumerate}
 \item własny \textbf{App extends RoboApplication} gdzie nadpisujemy \textbf{\#addApplicationModules}
 \pause \item własny \textbf{SzczecinModule extends AbstractAndroidModule}, który dodajemy powyżej
 \pause \item dodanie \verb|<application android:name=".App"| dla naszej aplikacji w \textbf{AndroidManifest.xml}
 \pause \item zamiana \verb|MyActivity extends Activity| na: \verb|MyActivity extends RoboActivity|
\end{enumerate}
\end{frame}


\begin{frame}[fragile]\frametitle{Zadanie: Kroczki do przodu}
\begin{itemize}
 \item podpinamy \textbf{RoboGuice}
 \item zapisujemy \textbf{imię użytkownika} w \textbf{SharedPreferences}
 \item podpinamy \textbf{res/menu/menu.xml} (1 element menu, o nazwie 'settings') (tip: robi się to w \textbf{onCreateOptionsMenu()})
\end{itemize} 

tip:
\begin{verbatim}
bindConstant()
.annotatedWith(SharedPreferencesName.class)
.to("pl.project13");
\end{verbatim}
\end{frame}





% \end{document}
